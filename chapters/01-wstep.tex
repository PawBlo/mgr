
\chapter{Wstęp}

% Wstęp do pracy powinien zawierać następujące elementy:
% \begin{itemize}
%     \item krótkie uzasadnienie podjęcia tematu; 
%     \item cel pracy (patrz niżej); 
%     \item zakres (przedmiotowy, podmiotowy, czasowy) wyjaśniający, w jakim rozmiarze praca będzie realizowana; 
%     \item ewentualne hipotezy, które autor zamierza sprawdzić lub udowodnić; 
%     \item krótką charakterystykę źródeł, zwłaszcza literaturowych; 
%     \item układ pracy (patrz niżej), czyli zwięzłą charakterystykę zawartości poszczególnych rozdziałów; 
%     \item ewentualne uwagi dotyczące realizacji tematu pracy np.~trudności, które pojawiły się w trakcie 
%     realizacji poszczególnych zadań, uwagi dotyczące wykorzystywanego sprzętu, współpraca z firmami zewnętrznymi. 
% \end{itemize}

Gałaź nauki zajmujące się "Model Checking" to obecnie bardzo szybko rozwijająca się gałąź nauki.
Jest to spowodowane licznymi zastosowaniami w których można skorzystać z narzędzii które dostarcza 
werfikacji modeli. Obecnie zastosowania jakie możemy znaleźć między innymi to
\begin{itemize}
    \item gry planszowe  ~\cite{Schlingloff}% Modeling and Analysis of Board Games
    \item planowanie misji dla autonomicznych agentów ~\cite{Gu}
    \item analiza i dowodzenia poprawności kodów źródłowych  ~\cite{Amazon}.
\end{itemize}

Celem który przyświecał podczas tworzenia tej pracy była próba stworzenia alternatywnych sposobów werfikacji 
modeli, skuteczniejszych od tych obecnie istniejących i wyznaczającyh standardy. Laureatka Nagrody 
Nobla Françoise Barré-Sinoussi w dziedzinie medycny powiedziała
\begin{quote}
    We are not making science for science. We are making science for the benefit of humanity.
    \end{quote}


Ciekawym obszarem jest analiza i dowodzenie poprawności kodów programistycznych. Zwłaszcza w obecnych czasach
gdy ilość kodu wytrzwarzana przez progmiastów z całego świata jest bardzo duża. Rok do roku liczba projektów programistycznych
rozpoczętych przez programistów wzrasta w dynamicznym tempie. Jednym z powodów jest pojawienie się ostatio
 modeli sztucznej inteligencji potrafiących
automatycznie generować oraz odpowiadać na pytania. Więcej o statystykach można poczytać w  ~\cite{Github}.

W tej pracy główny nacisk został położony na anlizę gier planszowych. Będziemy próbowali odpowiedzieć 
na pytanie czy i w jakim zakresie algorytmy uczenia ze wzmocnienieum i/lub algorytmy heurystyczne są w stanie 
polepszyć narzędzie typu "model checker". Narzędzie te umożliwiają automatyczną werfikację modeli.
Narzędzie te przyjmują na wejściu parę składającą się z modelu oraz formuły która będzie badana dla danego modelu.
Obecnie istnieje kilka tego typu narzędzi których opis znajdzie się w dalszej cześci pracy.

% \end{quote}

